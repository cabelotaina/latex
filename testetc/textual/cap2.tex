% Capitulo 2 - QEE e Disturbios elétromagneticos
\chapter{\capdois}\label{qeeDIS}
\section{QEE e EE}\label{qeeEE}
\par 
Mobilizar a sociedade para o uso eficiente da energia elétrica, combatendo o seu desperdício é a missão estratégica do PROCEL (Programa Nacional de Conservação de Energia Elétrica). A economia de energia elétrica proporciona inúmeras vantagens, entre elas a liberação de recursos para outras áreas e a contribuição para a preservação do meio ambiente [ELE10].
\par
Os fatores que influenciam na Qualidade da Energia Elétrica (QEE) podem ser originados tanto nas concessionárias como nos sistemas consumidores [MEL08]. Estes distúrbios podem ser gerados por fenômenos naturais, por operações da concessionária ou pelos próprios consumidores. Da mesma forma que uma maior demanda em horário de ponta causa perturbações e desequilíbrio na rede, além de penalizações por parte da ANEEL (Agência Nacional de Energia Elétrica) caso o consumo ultrapasse o limite contratado. 
\par
Contextualizando a QEE no cenário industrial, torna-se importante que as indústrias tenham um gerenciamento eficaz das demandas energéticas consumidas diariamente, evitando tanto o desperdício de energia quanto a diminuição da qualidade da mesma. Tendo em vista que a QEE pode chegar ao consumidor com um excelente padrão de qualidade, o mau gerenciamento, ou até mesmo problemas decorrentes da própria infraestrutura da empresa, podem implicar na diminuição da qualidade da energia consumida no processo industrial. 
\par
Conforme [SIL09] e [SOL04], as indústrias são peças importantes no contexto estudado, devido ao grande consumo de energia elétrica necessário em seus processos de produção. Em função disto, [SOL04] define como pontos relevantes para a pesquisa dentro desta área:

\section{Distúrbios Elétromagneticos}\label{dis}