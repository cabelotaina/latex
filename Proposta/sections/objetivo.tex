\chapter{Objetivo} O objetivo deste estudo está na concepção, validação e
avaliação de um sistema completo para reconhecimento de placas de veículos.
Isso é possível através da aplicação de conceitos que envolvem a criação de
sistemas embarcados, \textit{System-on-Chip}, processamento de imagem, redes
neurais e linux.  Este trabalho será baseado na tese de mestrado de Pacheco
\cite{PACHECO}, que desenvolveu uma nova metodologia de localização de
regiões
 candidatas em imagens digitais utilizando arquiteturas
reconfiguráveis. A meta
 é utilizar o modelo já implementado em VHDL para
criar um sistema completo de OCR.  Este trabalho de conclusão de curso tem como
finalidade fazer a integração e
 validação de um sistema que faça a captura de
imagem, extraia as regiões candidatas da placa contidas em
 uma imagem
digital, extrair os caracteres e faça o reconhecimento através da
 aplicação
de redes neurais.

A metodologia empregada para executar a tarefa de captura de imagem,
localização da placa, extração dos caracteres e reconhecimento 
faz uso
 de diferentes técnicas de processamento e análise de
imagens. Desta forma, o
 objetivo final deste trabalho é validar
primeiramente o OCR em software e
 posteriormente em \textit{hardware}.
 Todas as etapas
do modelo proposto serão validadas em \textit{Field Programmable Gate Array}
(FPGA) visando obter
 melhores desempenhos de processamento da imagem.  As
tarefas que executam as
 funções de extração de bordas, marcação dos pontos e
filtragem dos mesmos, serão executadas dedicadamente, sendo essa, uma das
principais características da arquitetura utilizada. 
 
