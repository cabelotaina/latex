\chapter{Introdução}

	A visão e a audição são os dois principais meios pelos quais os seres
	interpretam os sinais do mundo exterior \cite{LIM}. O interesse em
	métodos de processamento de imagens digitais decorre de duas áreas
	principais de aplicação: melhoria de informação visual para a
	interpretação humana e o processamento de dados de cenas para percepção
	automática através de máquinas.

Foi no século 20 que aconteceu uma das primeiras aplicações da utilização de
técnicas de processamento de imagens e tinha como principal objetivo melhorar
as imagens digitalizadas para jornais.  Essas digitalizações posteriormente
eram enviadas por meio de cabo submarino entre Londres e Nova Iorque. O tempo
necessário para esta transmissão era de uma semana e o sistema Bartlane de
transmissão de imagens por cabo submarino (como foi chamado) conseguiu reduzir
a transmissão para três horas.

Avanços expressivos na área vieram apenas com o advento dos computadores
digitais trinta décadas mais tarde \cite{GONZALES}.  Em 1964, fotos da lua enviadas pela
missão Ranger 7, foram processadas com o objetivo de corrigir vários tipos de
distorções inerentes à câmera utilizada. Estas técnicas serviram como base para
métodos de aprimoramento de realce e restauração de imagens de outros programas
espaciais posteriores como as expedições tripuladas da série Apollo \cite{GONZALES}.

Desde a década de 60 o processamento digital de imagem vem crescendo
substancialmente. Além das aplicações no programa espacial, técnicas de
processamento de imagens são atualmente utilizadas para resolver tarefas do
cotidiano.  Embora não relacionadas com freqüência, essas tarefas comumente
requerem métodos capazes de melhorar a informação visual para a análise e
interpretação humana.  Desta forma, dentro do campo de processamento de
imagens, a integração e validação de um sistema de reconhecimento de
caracteres é uma tarefa complexa e será melhor explicado no decorrer do TC.


