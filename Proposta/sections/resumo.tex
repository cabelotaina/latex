\chapter{resumo} Sistemas de processamento de imagens requerem um grande poder
computacional quando utilizadas em um processador de propósito geral.
Aplicações desse tipo utilizam pequenas partes do conjunto de instruções de um
processador. Este trabalho propõe a implementação de um sistema de reconhecimento
ótico de caracteres(OCR) em imagens digitais utilizando hardware
reprogramável com sistema de captura de imagens. Esta metodologia foi
desenvolvida por \cite{PACHECO} e encontra-se protegida pelos
direitos de propriedade intelectual sob o número 0000270607032603. O modelo
terá integração entre hardware e software em um sistema embarcado.  Esse
sistema pretende, além de aumentar o desempenho em relação a um sistema de OCR
tradicional, ter baixo consumo. Ambas implementaçãoes serão comparadas com o
objetivo de confrontar o desempenho e o consumo de energia com a execução dos
algoritmos em PC e FPGA XILINX Virtex-5 modelo XC5VLX110T®.



