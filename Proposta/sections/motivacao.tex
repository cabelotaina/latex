\chapter{Motivação}

Sistemas de processamento de imagem necessitam de grande poder computacional.
FPGA's podem oferecer esse processamento por criar \textit {hardware} dedicado
para resolver determinado tipo de tarefa. Sistemas de reconhecimento automático de
placas tornaram-se a solução em Sistemas Inteligentes de Transportes (SIT), em
que a visão artificial e padrões de reconhecimento são extensivamente aplicados
\cite{GUANGZHI}.

A identificação de veículos através do reconhecimento da placa veicular já era
usada na década de 50, tendo como objetivo o estudo do tempo de duração de
viagens entre origem e destino de veículos automotores. Os métodos iniciais
utilizavam observadores para anotar em papel, ou fitas gravadas, as placas dos
veículos e os tempos correspondentes à viagem, que depois eram comparadas
manualmente.

Desta forma, a necessidade de analisar imagens é de extrema importância para
automatizar processos. Assim, pode-se citar exemplos de aplicações de um
sistema de identificação automática de veículos no controle do tráfego; no
reconhecimento de veículos em situação irregular; no conrole de pedágios,
estacionamentos, aeroportos e intersecções; na administração de entradas
privativas; no pagamento automático de bilhetes.


