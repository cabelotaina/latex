\documentclass[12pt,a4paper,oneside]{article}
\usepackage[utf8]{inputenc}
\usepackage[portuguese]{babel}
\usepackage[T1]{fontenc}
\usepackage{amsmath}
\usepackage{amsfonts}
\usepackage{amssymb}
\usepackage{graphicx}
\usepackage[left=3cm,right=2cm,top=3cm,bottom=3cm]{geometry}
\usepackage{}

%% Cabeçalho e rodapé
% Controlar os cabeçalhos e rodapés
\usepackage{fancyhdr}
% Usar os estilos do pacote fancyhdr
\pagestyle{fancy}
\fancypagestyle{plain}{\fancyhf{}}
% Limpar os campos do cabeçalho atual
\fancyhead{}
% Número da página do lado esquerdo [L] nas páginas ímpares [O] e do lado direito [R] nas páginas pares [E]
\fancyhead[RO]{\thepage}
% Limpar os campos do rodapé
\fancyfoot{}
% Omitir linha de separação entre cabeçalho e conteúdo
\renewcommand{\headrulewidth}{0pt}
% Omitir linha de separação entre rodapé e conteúdo
\renewcommand{\footrulewidth}{0pt}
\author{Luana de Freitas}
\title{Testelayout}
\pagenumbering{arabic}

%%%%%%%%%%%%%%%%%%%%%%%%%%%%%%%%%%%%%%%%%%%%%%%%%%%%%%%%%%%%%%%%%%%%%%%%%%%%%%

\begin{document}
\pagestyle{fancy}
\noindent{\textsc{\textbf{Introdução}}}
\\
\par 
Mobilizar a sociedade para o uso eficiente da energia elétrica, combatendo o seu desperdício é a missão estratégica do PROCEL (Programa Nacional de Conservação de Energia Elétrica). A economia de energia elétrica proporciona inúmeras vantagens, entre elas a liberação de recursos para outras áreas e a contribuição para a preservação do meio ambiente [ELE10].
\par
Os fatores que influenciam na Qualidade da Energia Elétrica (QEE) podem ser originados tanto nas concessionárias como nos sistemas consumidores [MEL08]. Estes distúrbios podem ser gerados por fenômenos naturais, por operações da concessionária ou pelos próprios consumidores. 
Da mesma forma que uma maior demanda em horário de ponta causa perturbações e desequilíbrio na rede, além de penalizações por parte da ANEEL (Agência Nacional de Energia Elétrica) caso o consumo ultrapasse o limite contratado.
\par
Contextualizando a QEE no cenário industrial, torna-se importante que as indústrias tenham um gerenciamento eficaz das demandas energéticas consumidas diariamente, evitando tanto o desperdício de energia quanto a diminuição da qualidade da mesma. 
Tendo em vista que a QEE pode chegar ao consumidor com um excelente padrão de qualidade, o mau gerenciamento, ou até mesmo problemas  decorrentes da própria infraestrutura da empresa, podem implicar na diminuição da qualidade da energia consumida no processo industrial.

\end{document}
